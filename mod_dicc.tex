\documentclass{article}
\usepackage[spanish]{babel}
\usepackage{xifthen}
\usepackage{mathtools}
\usepackage{enumitem}

% ================= MACROS ================ %

\newcommand*{\IntSeExplica}[1]{%
    \textbf{se explica con}: \textsc{#1}%
}

\newcommand*{\IntGenero}[1]{%
    \textbf{géneros}: \texttt{#1}%
}

\newcommand*{\IntTitulo}[1]{%
    \noindent\textbf{\large #1}%
}

\newcommand*{\IntParametros}{%
    \textbf{parámetros formales}%
}

\newcommand*{\RepSRC}[2]{%
    \texttt{#1}  \textbf{se representa con} \texttt{#2}%
}

\newcommand{\IntFuncion}[8]{%
    \textsc{#1}(#2)\ifthenelse{\isempty{#3}}{\empty}{ $\to$ $res$ : \texttt{#3}}%
    \newline\indent\textbf{Pre} $\equiv$ \{#4\}%
    \newline\indent\textbf{Post} $\equiv$ \{#5\}%
    \newline\indent\textbf{Complejidad:} #6%
    \newline\indent\textbf{Descripción:} #7%
    \ifthenelse{\isempty{#8}}{\empty}{\newline\indent\textbf{Aliasing:} #8}%
}

\newcommand*{\IntIn}[2]{%
    \textbf{in} \ensuremath{#1} : \texttt{#2}%
}

\newcommand*{\IntOut}[2]{%
    \textbf{out} \ensuremath{#1} : \texttt{#2}%
}

\newcommand*{\IntInOut}[2]{%
    \textbf{in/out} \ensuremath{#1} : \texttt{#2}%
}

\newcommand*{\obseq}{\ensuremath{=_{\mathrm{obs}}}\thinspace}

\newcommand*{\wh}[1]{\ensuremath{\widehat{#1}}}

\newcommand*{\tadf}[2]{#1\ifthenelse{\isempty{#2}}{\empty}{(\thinspace#2\thinspace)}}

\DeclarePairedDelimiter\abs{\lvert}{\rvert}

\newenvironment{Interfaz}{%
    \noindent\textbf{\Large Interfaz}%
    \parskip=2ex%
}{}

\newenvironment{Representacion}{%
    \noindent\textbf{\Large Representación}%
}{}

\newenvironment{Algoritmos}{%
    \noindent\textbf{\Large Algoritmos}%
}{}

\newcommand*{\dsq}{diccSeq(secu($\alpha$),\thinspace$\sigma$)}
\newcommand*{\sq}{secu($\alpha$)}
\newcommand*{\sth}[1]{\ensuremath{\Theta(\thinspace#1\thinspace)}}
% ========================================= %

\begin{document}

\section{Módulo Diccionario Secuencias (secu($\alpha$), $\sigma$)}

\begin{Interfaz}

    \IntParametros
    \setlength{\hangindent}{30pt}
    \newline \IntGenero{$\alpha,\thinspace\sigma$}
    \newline \IntFuncion{$\bullet = \bullet$}{\IntIn{a_1}{$\alpha$}, \IntIn{a_2}{$\alpha$}}{bool}{true}{$res$ \obseq ($a_1 = a_2$)}{}{}{}

    \IntSeExplica{Diccionario($\kappa$,\thinspace$\sigma$)}

    \IntGenero{\dsq}

    \IntTitulo{Operaciones básicas de diccionario}

    \IntFuncion{Vacío}{}{\dsq}{true}{$res$ \obseq \tadf{vacío}{}}{$\Theta(1)$}{genera un diccionario vacío.}{}

    \IntFuncion{Definir}{\IntInOut{d}{\dsq}, \IntIn{k}{\sq}, \IntIn{s}{$\sigma$}}{}{\wh{d} \obseq $d_0$}{\wh{d} \obseq \tadf{definir}{$k$, $s$, $d_0$}}{\sth{1}}{define la clave $k$ con el significado $s$ en el diccionario.}{referencia copia, etc COMPLETAR}

    \IntFuncion{Definido?}{\IntIn{d}{\dsq}, \IntIn{k}{\sq}}{bool}{true}{$res$ \obseq \tadf{def?}{$k$, \wh{d}}}{\sth{\abs{k}}}{devuelve \texttt{true} si la clave $k$ está definida en el diccionario.}{la clave se pasa por copia, COMPLETAR.}

    \IntFuncion{Obtener}{\IntIn{d}{\dsq}, \IntIn{k}{\sq}}{$\sigma$}{\tadf{def?}{$k$, \wh{d}}}{$res$ \obseq \tadf{obtener}{$k$, \wh{d}}}{\sth{\abs{k}}}{devuelve el significado de la clave $k$.}{COPIA REFERENCIA? COMPLETAR}

    \IntFuncion{Borrar}{\IntInOut{d}{\dsq}, \IntIn{k}{\sq}}{}{(\wh{d} \obseq $d_0$) $\land$ \tadf{def?}{$k$, \wh{d}}}{\wh{d} \obseq \tadf{borrar}{$k$, $d_0$}}{$\Theta(\abs{k})$}{elimina la clave $k$ y su significado del diccionario.}{}

    \IntFuncion{Claves}{\IntIn{d}{\dsq}}{conj(\sq)}{true}{$res$ \obseq \tadf{claves}{\wh{d}}}{\sth{1}}{devuelve el conjunto de claves del diccionario.}{iterador a conjunto para borrar en long($k$)}

    \IntFuncion{Copiar}{\IntIn{d}{\dsq}}{\dsq}{true}{$res$ \obseq \wh{d}}{$\Theta$}{genera una copia del diccionario.}{}


\end{Interfaz}

\end{document}
