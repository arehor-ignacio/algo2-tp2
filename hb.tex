\documentclass{article}
\usepackage[spanish]{babel}
\usepackage{xifthen}
\usepackage{mathtools}
\usepackage{enumitem}
\usepackage{algorithmicx, algpseudocode, algorithm}

% ================= MACROS ================ %

\newcommand*{\IntSeExplica}[1]{%
    \textbf{se explica con}: \textsc{#1}%
}

\newcommand*{\IntGenero}[1]{%
    \textbf{géneros}: \texttt{#1}%
}

\newcommand*{\IntTitulo}[1]{%
    \noindent\textbf{\large #1}%
}

\newcommand*{\IntParametros}{%
    \textbf{parámetros formales}%
}

\newcommand*{\RepSRC}[2]{%
    \texttt{#1}  \textbf{se representa con} \texttt{#2}%
}

\newcommand{\IntFuncion}[8]{%
    \textsc{#1}(#2)\ifthenelse{\isempty{#3}}{\empty}{ $\to$ $res$ : \texttt{#3}}%
    \newline\indent\textbf{Pre} $\equiv$ \{#4\}%
    \newline\indent\textbf{Post} $\equiv$ \{#5\}%
    \newline\indent\textbf{Complejidad:} #6%
    \newline\indent\textbf{Descripción:} #7%
    \ifthenelse{\isempty{#8}}{\empty}{\newline\indent\textbf{Aliasing:} #8}%
}

\newcommand*{\IntIn}[2]{%
    \textbf{in} \ensuremath{#1} : \texttt{#2}%
}

\newcommand*{\IntOut}[2]{%
    \textbf{out} \ensuremath{#1} : \texttt{#2}%
}

\newcommand*{\IntInOut}[2]{%
    \textbf{in/out} \ensuremath{#1} : \texttt{#2}%
}

\newcommand*{\obseq}{\ensuremath{=_{\mathrm{obs}}}\thinspace}

\newcommand*{\wh}[1]{\ensuremath{\widehat{#1}}}

\newcommand*{\tadf}[2]{#1\ifthenelse{\isempty{#2}}{\empty}{(\thinspace#2\thinspace)}}

\DeclarePairedDelimiter\abs{\lvert}{\rvert}

\newenvironment{Interfaz}{%
    \noindent\textbf{\Large Interfaz}%
    \parskip=2ex%
}{}

\newenvironment{Representacion}{%
    \noindent\textbf{\Large Representación}%
}{}

\newenvironment{Algoritmos}{%
    \noindent\textbf{\Large Algoritmos}%
}{}

\newcommand*{\dsq}{diccSeq(secu($\alpha$),\thinspace$\sigma$)}
\newcommand*{\sq}{secu($\alpha$)}
\newcommand*{\sth}[1]{\ensuremath{\Theta(\thinspace#1\thinspace)}}
\newcommand*{\tuple}[1]{\ensuremath{\langle\thinspace#1\thinspace\rangle}}
% ========================================= %

\begin{document}

\section{Módulo Tabla}

\begin{Interfaz}

    \IntSeExplica{Tabla}

    \IntGenero{tabla}

    \IntTitulo{Operaciones básicas de tabla}

    \IntFuncion{Nueva}{\IntIn{cs}{conj(nombre\_campo)}, \IntIn{k}{nombre\_campo}}{tabla}{$k$ $\in$ $cs$}{$res$ \obseq \tadf{nueva}{$cs$, $k$}}{}{Crea una nueva tabla con los campos del conjunto $cs$, donde $k$ es el nombre el campo clave.}{}

    \IntFuncion{Insertar}{\IntInOut{t}{tabla}, \IntIn{r}{registro}}{}{\wh{t} \obseq $t_0$ $\land$ \tadf{campos}{\wh{t}} \obseq \tadf{campos}{$r$}}{\wh{t} \obseq \tadf{insertar}{$t_0$, $r$}}{}{Agrega el registo $r$ a la tabla $t$. Si esta ya tiene un registro cuya clave sea igual a la clave de $r$, lo sobreescribe.}{}

    \IntFuncion{Campos}{\IntIn{t}{tabla}}{conj(nombre\_tabla)}{true}{$res$ \obseq \tadf{campos}{\wh{t}}}{\sth{1}}{devuelve el conjunto campos de la tabla.}{devuelve una referencia del conjunto de campos de la tabla.}

    \IntFuncion{Clave}{\IntIn{t}{tabla}}{nombre\_campo}{true}{$res$ \obseq \tadf{clave}{\wh{t}}}{\sth{1}}{devuelve el nombre del campo clave de la tabla.}{devuelve una referencia del campo clave de la tabla.}

    \IntFuncion{Registros}{\IntIn{t}{tabla}}{conj(registro)}{true}{$res$ \obseq \tadf{registros}{\wh{t}}}{\sth{1}}{devuelve el conjunto de registros de la tabla.}{el conjunto de registros se devuelve por referencia.}

    \IntFuncion{Borrar}{\IntInOut{t}{tabla}, \IntIn{v}{valor}}{}{\wh{t} \obseq $t_0$}{\wh{t} \obseq \tadf{borrar}{$t_0$, $v$}}{}{Borra de la tabla $t$ aquel registro cuya clave sea igual a $v$.}{}

\end{Interfaz}

\begin{Representacion}

    \RepSRC{tabla}{tbl} 

    donde \texttt{tbl} es \texttt{tupla} $<$$valoresEnCampo$: \texttt{diccSeq(nombre\_campo, diccSeq(valor, conj(itConj(registro))))},
    \newline\indent$clave$: \texttt{nombre\_campo},
    \newline\indent$registros$: \texttt{conj(registro)}$>$

    Rep: diccSeq(valor, registro) $\to$ bool

    Rep(t) $\equiv$ true $\iff$ para todo v: valor $\land$ def?(v, d) $\to$ (luego) campos(obtener(v, d)) = campos y k = clave(obtener(v, d)).


\end{Representacion}

\begin{Algoritmos}

    \begin{algorithm}[H]{\textbf{iNueva}(\IntIn{cs}{conj(nombre\_campo)}, \IntIn{k}{nombre\_campo}) $\to$ $res$ : tbl}
        \begin{algorithmic}[1]
        \State $itCampos$ $\gets$ crearIt($cs$)                 \Comment{\sth{1}}
        \State $dic$ $\gets$ Vacío()
        \While{HaySiguiente?($itCampos$)}
        \State Definir($dic$, Siguiente($itCampos$), Vacío())   \Comment{$\Theta({\abs{c}})$}
        \EndWhile
        \State $regs$ $\gets$ Vacío()
        \State $res$ $\gets$ \tuple{dic, regs, k}               \Comment{\sth{1}}

        \medskip
        \State \underline{Complejidad:} \sth{n*\abs{c}}
        \end{algorithmic}
    \end{algorithm}

\end{Algoritmos}

\end{document}
